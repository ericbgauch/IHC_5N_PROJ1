
\documentclass[
    % -- opções da classe memoir --
    12pt,               % tamanho da fonte
    openany,          	% capítulos começam em páginas ímpar
    twoside,            % para impressão em verso e anverso. Oposto a oneside
    a4paper,            % tamanho do papel. 
    % -- opções da classe abntex2 --
    %chapter=TITLE,     % títulos de capítulos convertidos em letras maiúsculas
    %section=TITLE,     % títulos de seções convertidos em letras maiúsculas
    %subsection=TITLE,  % títulos de subseções convertidos em letras maiúsculas
    %subsubsection=TITLE,% títulos de subsubseções convertidos em letras maiúsculas
    % -- opções do pacote babel --
    brazil              % o último idioma é o principal do documento
    ]{abntex2}

% ---
% Pacotes básicos 
% ---
\usepackage{palatino}           % Usa a fonte Latin Modern
\usepackage[T1]{fontenc}        % Selecao de codigos de fonte.
\usepackage[utf8]{inputenc}     % Codificacao do documento (conversão automática dos acentos)
\usepackage{lastpage}           % Usado pela Ficha catalográfica
\usepackage{indentfirst}        % Indenta o primeiro parágrafo de cada seção.
\usepackage{color}              % Controle das cores
\usepackage{graphicx}           % Inclusão de gráficos
\usepackage{microtype}          % para melhorias de justificação
\usepackage{graphicx}
\graphicspath{ {images/} }
% ---
        
% ---
% Informações de dados para CAPA e FOLHA DE ROSTO
% ---
\titulo{Definição do Projeto de Interação Humano-Computador}
\autor{
    Eric Bueno Gauch
    \and
    Bruno Estima Correia Milanesi Castanheira de Souza
    \and
    Filippi Luigi Di Pipi
    \and
    Guilherme Gatto Gonçalves da Silva}
\local{Brasil}
\data{2017, v-1.0.0}
\instituicao{%
  Universidade Presbiteriana Mackenzie -- UPM
  \par
  Faculdade de Computação e Informática
  \par
  Programa de Graduação em Ciência da Computação}
\tipotrabalho{Trabalho Acadêmico}
% O preambulo deve conter o tipo do trabalho, o objetivo, 
% o nome da instituição e a área de concentração 
\preambulo{Definição do projeto da disciplina de Interação Humano-Computador}
% ---

% ---
% Configurações de aparência do PDF final

% alterando o aspecto da cor azul
\definecolor{blue}{RGB}{41,5,195}

% informações do PDF
\makeatletter
\hypersetup{
        %pagebackref=true,
        pdftitle={\@title}, 
        pdfauthor={\@author},
        pdfsubject={\imprimirpreambulo},
        pdfcreator={LaTeX with abnTeX2},
        pdfkeywords={abnt}{latex}{abntex}{abntex2}{trabalho acadêmico}, 
        colorlinks=true,            % false: boxed links; true: colored links
        linkcolor=black,             % color of internal links
        citecolor=blue,             % color of links to bibliography
        filecolor=magenta,              % color of file links
        urlcolor=blue,
        bookmarksdepth=4
}
\makeatother
% --- 

% O tamanho do parágrafo é dado por:
\setlength{\parindent}{1.3cm}

% Controle do espaçamento entre um parágrafo e outro:
\setlength{\parskip}{0.2cm}  % tente também \onelineskip

% ---
% compila o indice
% ---
\makeindex
% ---

% ----
% Início do documento
% ----
\begin{document}

% Retira espaço extra obsoleto entre as frases.
\frenchspacing 

% ----------------------------------------------------------
% ELEMENTOS PRÉ-TEXTUAIS
% ----------------------------------------------------------
% \pretextual

% ---
% Capa
% ---
\imprimircapa
% ---

% ---
% Folha de rosto
% (o * indica que haverá a ficha bibliográfica)
% ---
\imprimirfolhaderosto*
% ---

% ---
% inserir o sumario
% ---
\clearpage
\pdfbookmark[0]{\contentsname}{toc}
\tableofcontents*
\cleardoublepage
% ---

% ----------------------------------------------------------
% ELEMENTOS TEXTUAIS
% ----------------------------------------------------------
\textual

% ----------------------------------------------------------
% CAPÍTULO
% ----------------------------------------------------------
\chapter{Integrantes do Grupo}
% ----------------------------------------------------------

\begin{center}
\begin{tabular}{ | l | l | l | p{5cm} |}
\hline
NOME & TIA \\ \hline
Bruno Estima Correia Milanesi Castanheira de Souza
& 4152272-9 \\ \hline
Eric Bueno Gauch & 4151051-8 \\ \hline
Filippi Luigi Di Pipi & 3143893-8 \\ \hline
Guilherme Gatto Gonçalves da Silva & 4150246-9 \\ \hline
\end{tabular}
\end{center}

% ----------------------------------------------------------
% CAPÍTULO
% ----------------------------------------------------------
\chapter{Assunto a ser Coberto}
% ----------------------------------------------------------

Quase sempre a matemática é a grande vilã nos boletins escolares. Ao lado de
outras disciplinas das ciências exatas como, por exemplo, física e química, a
matemática se torna uma verdadeira armadilha para o bom desempenho. Muitos
acreditam que isso se deve à forma como ela é desenvolvida, ou seja, como ela
está no currículo para ser cumprida. Além de outras questões, bem como a
interação com algo concreto, algo que a criança possa relacionar com a
aplicação.

Acredita-se que ensinando aos alunos uma matemática `viva' e os ajudando a
estabelecer uma relação entre a disciplina e seus cotidianos é uma das formas
de diminuir a rejeição pela matemática, pois deixa o aluno frente à uma
matemática mais interessante.

% ----------------------------------------------------------
% CAPÍTULO
% ----------------------------------------------------------
\chapter{Ideia}
% ----------------------------------------------------------

Tendo em vista o grande desinteresse que crianças entre 7 a 10 anos têm na
matemática, o jogo facilita o ensino das quatro operações básicas da matemática
(adição, subtração, multiplicação e divisão) de forma divertida e interativa.
Focado em um público infantil, o jogo tem um tema robótico, cenário colorido e
uma interface simples e direta, características que prendem a atenção do
jogador e estimula a imersão no mundo dos números.

Ao entrar no jogo, duas mãos de aparência robótica são apresentadas com alguns
dedos levantados e um dos símbolos matemáticos básicos. Para que o jogador
tenha uma boa posição no \textit{ranking}, é necessário acertar o máximo de
operações possíveis no menor tempo.

Com uma jogabilidade simples e intuitiva, o jogador deve analisar a quantidade
de dedos levantados nas mãos apresentadas na tela e realizar a operação
matemática solicitada. Por fim, o jogador deverá selecionar entre as mãos
apresentadas abaixo aquela(s) que corresponderm ao resultado da operação.

% ----------------------------------------------------------
% CAPÍTULO
% ----------------------------------------------------------
\chapter{\textit{Storyboard}}
% ----------------------------------------------------------

\includegraphics[scale=0.3]{storyboard}

% ----------------------------------------------------------
% CAPÍTULO
% ----------------------------------------------------------
\chapter{Requisitos Funcionais e Não Funcionais}
% ----------------------------------------------------------

% ----------------------------------------------------------
% SEÇÃO
% ----------------------------------------------------------
\section{Requisitos Funcionais}
% ----------------------------------------------------------

\begin{description}
\item[$\cdot$] Receber o nome do usuário
\item[$\cdot$] Efetuar as 4 operações básicas
\item[$\cdot$] Calcular a pontuação do usuário
\item[$\cdot$] Guardar as pontuações e os respectivos nomes em um \textit{ranking}
\end{description}

% ----------------------------------------------------------
% SEÇÃO
% ----------------------------------------------------------
\section{Requisitos Não Funcionais}
% ----------------------------------------------------------

\begin{description}
\item[$\cdot$] O \textit{software} deve ser fácil de usar e bastante intuitivo
\item[$\cdot$] Possuir uma interface bem colorida para despertar o interesse das crianças
\item[$\cdot$] Possuir boa performance, todos os movimentos devem ser realizados com fluidez
\end{description}

%----------------------------------------------------------
% CAPÍTULO
%----------------------------------------------------------

\chapter{Personas}
%----------------------------------------------------------

\begin{center} \begin{tabular}{ | l | p{5cm} | p{6cm} }
\hline Nome & Maria Clara Rios \\\hline
Idade & 9 anos\\ \hline
Ano escolar & 4º ano\\ \hline Usa
smartphones ou computadores? & Sim \\ \hline
Gosta de matemática? & Sim \\ \hline
Tem dificuldade com a matemática? & Um pouco \\ \hline
Como você gostaria que fosse a aula de matemática? & Com muitos jogos \\ \hline
\end{tabular} \end{center}

\begin{center} \begin{tabular}{ | l | p{5cm} | p{6cm} }
\hline Nome & João Pedro \\\hline
Idade & 8 anos\\ \hline
Ano escolar & 3º ano\\ \hline Usa
smartphones ou computadores? & Sim \\ \hline
Gosta de matemática? & Não, acho difícil \\ \hline
Tem dificuldade com a matemática? & Sim \\ \hline
O que você gosta de fazer? & Gosto muito de jogar bola \\ \hline
Como você gostaria que fosse a aula de matemática? & Poderia ser com brincadeiras \\ \hline
\end{tabular} \end{center}

\begin{center} \begin{tabular}{ | l | p{5cm} | p{6cm} }
\hline Nome & Miguel da Silva \\\hline
Idade & 6 anos\\ \hline
Ano escolar & 1º ano\\ \hline Usa
smartphones ou computadores? & Uso o celular do meu pai \\ \hline
Gosta de matemática? & Sim \\ \hline
Tem dificuldade com a matemática? & Um pouco \\ \hline
O que você gosta de fazer? & Gosto de jogar video game \\ \hline
Como você gostaria que fosse a aula de matemática? & Com jogos de tabuleiro \\ \hline
\end{tabular} \end{center}

\end{document}
